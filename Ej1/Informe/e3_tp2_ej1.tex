\documentclass[../../e3_tp2_main.tex]{subfiles}

\begin{document}
\section{Ejercicio 1}

En esta secci\'on se implementaron compuertas \textit{not} a partir de transistores BJT (de sus siglas en ingl\'es, \textit{bipolar junction transistor}). Particularmente, se decidi\'o utilizar BJTs de tipo NPN, formando circuitos de dos familias distintas:

\begin{itemize}
	\item RTL (\textit{resistor-transistor logic}): utiliza resistencias en la malla de entrada y transistores como \textit{switch}.
	\item TTL (\textit{transistor-transistor logic}): utiliza transistores tanto para el \textit{switching} como para la amplificaci\'on.
\end{itemize}

De estas definiciones surge que en TTL se requiere mayor cantidad de transistores por compuerta, pero se disipar\'a menos potencia en resistencias pues las corrientes ser\'an menores. Para comparar otros factores de su funcionamiento, se armaron los siguientes circuitos:

\begin{figure}[H]
	\centering
		\scalebox{1}{
		\begin{circuitikz}
	 	\draw 
 			(3.5,1) node[npn] (npn) {}
		 	(npn.B) to [R, l_=15K$\Omega$, -o] (0.5,1) node[left]{$V_{IN}$}	
 			(npn.E) to [short] (3.5,-0.5) node[ground]{}
 		 	(npn.C) to [R, l_=1K$\Omega$, -o] (3.5,3.5) node[above]{+5V}
 		 	(3.5,1.5) to [short, *-o] (4,1.5) node[right]{$V_{OUT}$} 	
 		;\end{circuitikz}
 		}
	\caption{Compuerta \textit{not} RTL}
	\label{fig:1-rtl}
\end{figure}

\begin{figure}[H]
	\centering
		\scalebox{1}{
		\begin{circuitikz}
	 	\draw 
 			(3.5,1) node[npn] (npn) {}
 			(npn.E) to [short] (3.5,-0.5) node[ground]{}
 		 	(npn.C) to [R, l_=1K$\Omega$, -o] (3.5,3.5) node[above]{+5V}
 		 	(3.5,1.5) to [short, *-o] (4,1.5) node[right]{$V_{OUT}$} 	
 		 	
			(1.5,1) node[npn, rotate=-90] (npn2){} 	
			(npn2.C) to [short] (npn.B)
			(npn2.E) to [short, -o] (0.5,1) node[left]{$V_{IN}$}
			(npn2.B) to [R, l_=4.7K$\Omega$, -o] (1.5, 3.5) node[above]{+5V}	 	
 		 	
 		;\end{circuitikz}
 		}
	\caption{Compuerta \textit{not} TTL}
	\label{fig:1-ttl}
\end{figure}


Se utilizaron transistores BC547 y resistencias de metal-film, y los circuitos se armaron en una \textit{printed circuit board}. \par

\subsection{An\'alisis de resultados}

Las mediciones se realizaron el circuito con ondas cuadradas de 5V de amplitud, con el nivel bajo en 0V. Las tensiones necesarias para calcular el margen de ruido se midieron utilizando el modo XY del osciloscopio con permanencia infinita, mientras que para los tiempos el modo \textit{single}. Todas las mediciones se realizaron en dos condiciones para cada compuerta: en vac\'io, y con un capacitor de 1nF como carga.\par

Los resultados de las mediciones se encuentran en las tablas \ref{table:1-mediciones-rtl} y \ref{table:1-mediciones-ttl}. Las magnitudes medidas fueron:

\begin{itemize}
	\item HLIV (\textit{high-level input voltage}): m\'axima tensi\'on de entrada con la cual la pendiente de $V_{IN}(V_{OUT})=-1$, es decir, la m\'inima con la que se interpreta un 1 en la entrada.
	\item LLIV (\textit{low-level input voltage}): m\'inima tensi\'on de entrada con la cual la pendiente de $V_{IN}(V_{OUT})=-1$, es decir, la m\'axima con la que se interpreta un 0 en la entrada.
	\item HLOV (\textit{high-level output voltage}): tensi\'on de salida cuando $V_{IN}=HLIV$, es decir, la m\'inima con la cual puede considerarse que hay un 1 en la salida.
	\item LLIV (\textit{low-level ouput voltage}): tensi\'on de salida cuando $V_{IN}=LLIV$, es decir, la m\'inima con la cual puede considerarse que hay un 1 en la salida.
	\item Noise margin: diferencia entre el \textit{high/low level input} y \textit{output} voltage, es decir rango de tensiones que pueden hallarse a la entrada pero no a la salida (pero que tienen comportamiento definido para la salida) para cada nivel l\'ogico. 
	\item PD (\textit{propagation delay}): tiempo que transcurre entre que la tensi\'on de entrada est\'a al 50\% entre su valor bajo y alto, y que lo opuesto ocurre en la salida.
	\item TT (\textit{transition time}): tiempo que tarda la salida en transicionar de un 10\% a un 90\% de la tensi\'on alta.
	\item Max. out. curr (\textit{maximum output current}): corriente de salida m\'axima, que por ser la carga puramente capacitiva, se calcul\'o como $i_c = C \cdot \frac{dV_C}{dt}$, obteniendo esta derivada a trav\'es de las funciones matem\'aticas del osciloscopio. \par
\end{itemize}


\begin{table}[H]
	\centering
	\begin{tabular}{|c|c|c|}
	\hline
        \textbf{RTL}           	& En vac\'io	& Con carga \\ \hline \hline
	HLIV (V)                    & 0.80              	& 0.8           \\ \hline
	LLIV (V)                    	& 0.57               & 0.57         \\ \hline
	HLOV (V)                  	& 4.80               & 4.80         \\ \hline
	LLOV (V)                   	& 0.27               & 0.27                              \\ \hline
	Noise margin H (V)     & 4.0             	& 4.0                              \\ \hline
	Noise margin L (V)	& 0.3			& 0.3		\\ \hline
	PD (H-L) (ns)		& 3855              & 4380                        \\ \hline
	PD (L-H) (ns) 		& 65                 & 140             \\ \hline
	TT (H-L) (ns)   		& 134               & 224                   \\ \hline
	TT (L-H) (ns)   		& 520                & 3860                   \\ \hline
	Max. out. curr. (mA)   & -                    & 51                     \\ \hline

	\end{tabular}
	
	\caption{Mediciones para la compuerta RTL}
	\label{table:1-mediciones-rtl}
\end{table}

\begin{table}[H]
	\centering
	\begin{tabular}{|c|c|c|c|c|}
	\hline
        \textbf{TTL}          	& En vac\'io	& Con carga \\ \hline \hline
	HLIV (V)                   	& 0.63             	& 0.63                       \\ \hline
	LLIV (V)                    	& 0.55             	& 0.55                      \\ \hline
	HLOV (V)                  	& 4.85             	& 4.85                \\ \hline
	LLOV (V)                   	& 0.21             	& 0.21                        \\ \hline
	Noise margin H (V)    	& 4.22              	& 4.22                                \\ \hline
	Noise margin L (V)	& 0.34		& 0.34				\\ \hline
	PD (H-L) (ns)		& 804               	& 1445                  \\ \hline
	PD (L-H) (ns) 		& 10               	& 15                   \\ \hline
	TT (H-L) (ns)   		& <13               & 70              \\ \hline
	TT (L-H) (ns)   		& 115               	& 3120                    \\ \hline
	Max. out. curr. (mA)   & -                  	& 16               \\ \hline

	\end{tabular}
	
	\caption{Mediciones para la compuerta TTL}
	\label{table:1-mediciones-ttl}
\end{table}

En primer lugar, debe tenerse en cuenta que las mediciones de los tiempos de transici\'on est\'an limitados por el \textit{rise time} m\'inimo del generador para ondas cuadradas, que es de 13ns\footnote{Informaci\'on obtenida de la \href{https://literature.cdn.keysight.com/litweb/pdf/5988-8544EN.pdf?id=187648}{\underline{hoja de datos}} del generador Agilent 33220A (consultado: 16/10/18).}, con lo cual para algunas mediciones s\'olo podemos afirmar que el valor obtenido es menor a este \'ultimo. \par

Se puede apreciar que, para ambos circuitos, la carga no afecta los valores de tensiones de entrada ni de salida que se consideran como 1 o 0, pero causa que se extiendan todos los tiempos, tanto de propagaci\'on como de transici\'on. Esto es consistente con el hecho de que la carga capacitiva limita la variaci\'on de tensi\'on para una misma corriente.\par

Se observa adem\'as que pr\'acticamente todos los par\'ametros medidos indican una mejor \textit{performance} de la compuerta TTL respuecto de la RTL. La principal ventaja de este \'ultimo tipo de tecnolog\'ia era el menor requerimiento en cantidad de transistores, lo cual era significante cuando estos componentes eran considerablemente m\'as caros que el resto del circuito. Actualmente, puesto que ya no es necesario el uso de transistores discretos sino que puede recurrirse a circuitos integrados y su precio ha disminuido, las compuertas RTL est\'an pr\'acticamente obsoletas. \par 

El \'unica medici\'on realizada que favorece a la compuerta RTL es la de m\'axima corriente de salida: una mayor corriente m\'axima indica un mayor \textit{fanout}. Esta es una de las principales limitaciones de las compuertas TTL.

\end{document}
