\documentclass[../../e3_tp2_main.tex]{subfiles}

\begin{document}
\chapter{}

En esta secci\'on se implementaron compuertas \textit{not} a partir de transistores BJT (de sus siglas en ingl\'es, \textit{bipolar junction transistor}). Particularmente, se decidi\'o utilizar BJTs de tipo NPN, formando circuitos de dos familias distintas:

\begin{itemize}
	\item RTL (\textit{resistor-transistor logic}): utiliza resistencias en la malla de entrada y transistores como \textit{switch}.
	\item TTL (\textit{transistor-transistor logic}): utiliza transistores tanto para el \textit{switching} como para la amplificaci\'on.
\end{itemize}

De estas definiciones surge que en TTL se requiere mayor cantidad de transistores por compuerta, pero se disipar\'a menos potencia en resistencias pues las corrientes ser\'an menores. Para comparar otros factores de su funcionamiento, se armaron los siguientes circuitos:

\begin{figure}[H]
	\centering
		\begin{circuitikz}
	 	\draw 
 			(3.5,1) node[npn] (npn) {}
		 	(npn.B) to [R, l_=15K$\Omega$, -o] (0.5,1) node[left]{$V_{IN}$}	
 			(npn.E) to [short] (3.5,-0.5) node[ground]{}
 		 	(npn.C) to [R, l_=1K$\Omega$, -o] (3.5,3.5) node[above]{+5V}
 		 	(3.5,1.5) to [short, *-o] (4,1.5) node[right]{$V_{OUT}$} 	
 		;\end{circuitikz}
	\caption{Compuerta \textit{not} RTL}
	\label{fig:1-rtl}
\end{figure}

\begin{figure}[H]
	\centering
		\begin{circuitikz}
	 	\draw 
 			(3.5,1) node[npn] (npn) {}
 			(npn.E) to [short] (3.5,-0.5) node[ground]{}
 		 	(npn.C) to [R, l_=1K$\Omega$, -o] (3.5,3.5) node[above]{+5V}
 		 	(3.5,1.5) to [short, *-o] (4,1.5) node[right]{$V_{OUT}$} 	
 		 	
			(1.5,1) node[npn, rotate=-90] (npn2){} 	
			(npn2.C) to [short] (npn.B)
			(npn2.E) to [short, -o] (0.5,1) node[left]{$V_{IN}$}
			(npn2.B) to [R, l_=4.7K$\Omega$, -o] (1.5, 3.5) node[above]{+5V}	 	
 		 	
 		;\end{circuitikz}
	\caption{Compuerta \textit{not} TTL}
	\label{fig:1-ttl}
\end{figure}


Se utilizaron transistores BC547 y resistencias de metal-film, y los circuitos se armaron en una \textit{printed circuit board}. \par

\section{An\'alisis de resultados}

Las mediciones se realizaron el circuito con ondas cuadradas de 5V de amplitud\todo{modelo del generador}, con el nivel bajo en 0V. Las tensiones necesarias para calcular el margen de ruido se midieron utilizando el modo XY del osciloscopio\todo{modelo del osciloscopio}, mientras que para los tiempos el modo \textit{single}. Todas las mediciones se realizaron en dos condiciones para cada compuerta: en vac\'io, y con un capacitor de 1nF como carga. Al cargar la compuerta, se pudo hacer adem\'as una medici\'on adicional: la 

Los resultados de las mediciones se encuentran en las tablas \ref{table:1-mediciones-rtl} y \ref{table:1-mediciones-ttl}. Las magnitudes medidas fueron:

\begin{itemize}
	\item HLIV (\textit{high-level input voltage}): m\'axima tensi\'on de entrada con la cual la pendiente de $V_{IN}(V_{OUT})=-1$, es decir, la m\'inima con la que se interpreta un 1 en la entrada.
	\item LLIV (\textit{low-level input voltage}): m\'inima tensi\'on de entrada con la cual la pendiente de $V_{IN}(V_{OUT})=-1$, es decir, la m\'axima con la que se interpreta un 0 en la entrada.
	\item HLOV (\textit{high-level output voltage}): tensi\'on de salida cuando $V_{IN}=HLIV$, es decir, la m\'inima con la cual puede considerarse que hay un 1 en la salida.
	\item LLIV (\textit{low-level ouput voltage}): tensi\'on de salida cuando $V_{IN}=LLIV$, es decir, la m\'inima con la cual puede considerarse que hay un 1 en la salida.
	\item Noise margin: diferencia entre el \textit{high/low level input} y \textit{output} voltage, es decir rango de tensiones que pueden hallarse a la entrada pero no a la salida (pero que tienen comportamiento definido para la salida) para cada nivel l\'ogico. 
	\item PD (\textit{propagation delay}): tiempo que transcurre entre que la tensi\'on de entrada est\'a al 50\% entre su valor bajo y alto, y que lo opuesto ocurre en la salida.
	\item TT (\textit{transition time}): tiempo que tarda la salida en transicionar de un 10\% a un 90\% de la tensi\'on alta.
	\item Max. out. curr (\textit{maximum output current}): corriente de salida m\'axima, que por ser la carga puramente capacitiva, se calcul\'o como $i_c = C \cdot \frac{dV_C}{dt}$, obteniendo esta derivada a trav\'es de las funciones matem\'aticas del osciloscopio. \par
\end{itemize}

Cabe destacar que el transition time medido en el caso de la compuerta TTL con carga, resultante en 10ns, no puede ser tomado como tal: debe considerarse que esto era respuesta a una se\~nal generada por un dispositivo que no asegura poder generar se\~nales con un rise time menor a 13ns, con lo cual este es el l\'imite inferior de lo que se puede medir en lo que a esta magnitud respecta.\par


\begin{table}[H]
	\centering
	\begin{tabular}{|c|c|c|}
	\hline
        RTL                          				& En vac\'io	& Con carga \\ \hline \hline
	HLIV (V)                          			&                      &                   \\ \hline
	LLIV (V)                          			&                      &                    \\ \hline
	HLOV (V)                          			&                      &                                    \\ \hline
	LLOV (V)                          			&                      &                               \\ \hline
	Noise margin H (V)                     	&                      &                                \\ \hline
	Noise margin L (V)				&			&				\\ \hline
	PD (H-L) ($\mu s$)	& 0.05               & 0.07                        \\ \hline
	PD (L-H) ($\mu s$) 	& 3.66               & 3.54             \\ \hline
	TT (H-L) ($\mu s$)   	& 0.08               & 0.20                   \\ \hline
	TT (L-H) ($\mu s$)   	& 0.56               & 3.14                   \\ \hline
	Max. out. curr. (mA)       	& -                    & 51                     \\ \hline

	\end{tabular}
	
	\caption{Mediciones para la compuerta RTL}
	\label{table:1-mediciones-rtl}
\end{table}

\begin{table}[H]
	\centering
	\begin{tabular}{|c|c|c|c|c|}
	\hline
        TTL                          				& En vac\'io & Con carga \\ \hline \hline
	HLIV (V)                          			&                      &                        \\ \hline
	LLIV (V)                          			&                      &                       \\ \hline
	HLOV (V)                          			&                      &                 \\ \hline
	LLOV (V)                          			&                      &                         \\ \hline
	Noise margin H (V)                     	&                      &                                \\ \hline
	Noise margin L (V)				&			&				\\ \hline
	PD (H-L) ($\mu s$)	& 0.05               & 0.07                  \\ \hline
	PD (L-H) ($\mu s$) 	& 3.66               & 3.54                   \\ \hline
	TT (H-L) ($\mu s$)   	& 0.08               & 0.20              \\ \hline
	TT (L-H) ($\mu s$)   	& 0.56               & 3.14                    \\ \hline
	Max. out. curr. (mA)       	& -                    & 51               \\ \hline

	\end{tabular}
	
	\caption{Mediciones para la compuerta TTL}
	\label{table:1-mediciones-ttl}
\end{table}



\end{document}
