\documentclass[../../e3_tp2_main.tex]{subfiles}

\begin{document}

\chapter{}

En esta secci\'on, estudiaremos la variaci\'on de los par\'ametros caracter\'isticos de una compuerta l\'ogica al cargar su salida, y al introducir capacitores de desacople. La compuerta que utilizaremos es una 74HC02, es decir una \textit{nor} de tecnolog\'ia CMOS.\par

Para observar el comportamiento del circuito al conectarle una carga, se arm\'o el siguiente circuito:

\begin{figure}[H]
	\centering
	\scalebox{0.8}{
	\begin{circuitikz}
	%\draw[help lines] (0,0) grid (7,8);
	\draw			
		
	(1.5,4) node[nor port](nor){}	
	
	(4,1) node[nand port](n1){}	
	(n1.in 1) to [short] (n1.in 2)
	
	(4,3) node[nand port](n2){}	
	(n2.in 1) to [short] (n2.in 2)
	
	(4,5) node[nand port](n3){}
	(n3.in 1) to [short] (n3.in 2)

	(4,7) node[nand port](n4){}
	(n4.in 1) to [short] (n4.in 2)
		
	(nor.out) to [short] (2,4)
	to [short] (2,7)
	to [short] (2.6,7) -| (n4.in 1)	
	
	(2,5) [short] to (2.6,5) -| (n3.in 1)	
	
	(2,4) to [short] (2, 3) 	
	to [short] (2.6,3)

	(2,3) to [short] (2, 1)
	to [short] (2.6, 1) -| (n1.in 1)
	
	(nor.in 1) to [short] (nor.in 2)
	(0.1,4) to [short, -o] (-.4,4) node[left]{$V_{IN}$}	
	
	(n4.out) to [R=1k$\Omega$] (6,7)
	to [full led] (7,7)
	to [short] (7.5,7)
	
	(n3.out) to [R=1k$\Omega$] (6,5)
	to [full led] (7,5)
	to [short] (7.5,5)
	
	(n2.out) to [R=1k$\Omega$] (6,3)
	to [full led] (7,3)
	to [short] (7.5,3)
	
	(n1.out) to [R=1k$\Omega$] (6,1)
	to [full led] (7,1)
	to [short] (7.5,1)
	
	(7.5,7) to [short] (7.5, -1) node[ground]{}
	
	(2,1) to [short] (2,-.5)
	to [short] (4,-.5)
	to [R=560$\Omega$] (6,-.5)
	to [full led] (7, -.5)
	to [short] (7.5, -.5)
	
	;\end{circuitikz}
	}
	\caption{Esquema del circuito}
\end{figure}

Las compuertas \textit{nand} utilizadas provinieron del integrado 74LS00, es decir que son de tecnolog\'ia TTL. Si bien estas tecnolog\'ias no son compatibles entre s\'i en principio, los rangos de tensiones que representan ceros o unos l\'ogicos coinciden en esta configuraci\'on (como se discuti\'o en el ejercicio 2). 

\begin{table}[H]
	\centering
	\begin{tabular}{|c|c|c|c|}
	\hline
                  & Vac\'io & Con carga & Con desacople \\ \hline \hline
	Propagaci\'on H-L &         &           &               \\ \hline
	Propagaci\'on L-H &         &           &               \\ \hline
	Rise time         &         &           &               \\ \hline
	Fall time         &         &           &               \\ \hline
	\end{tabular}
	\caption{Tiempos medidos (en microsegundos)}
\end{table}

\end{document}
