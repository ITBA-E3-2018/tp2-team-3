\documentclass[spanish]{article}

\makeatletter 
\def\input@path{{../../}} 
\makeatother
\input{e3_tp2_preamble.tex}

\begin{document}

\newgeometry{}

\input{e3_tp2_caratula.tex}

\newgeometry{left =1.5cm, right=1.5cm, top=1.5cm, bottom=2cm}	%texto ocupa mas ancho de pagina


\tableofcontents
\newpage






%hago backup del comando \input actual, y lo cambio para que cambie el path de buqueda de nuevos archivos
%esto va a afectar a los includes de este archivo y a los del archivo del subfile, pero solo cuando compilo desde aca. Si compilo desde el subfile este cambio no sucede entonces sigue andando.

\begin{multicols}{2}

\let\oldinput=\input 	
\def\input#1{\oldinput{Ej1/Informe/#1}}	
%agrego el archivo y despues restauro al comando original
\subfile{e3_tp2_ej1.tex}
\let\input=\oldinput

\let\oldinput=\input 	
\def\input#1{\oldinput{Ej2/Informe/#1}}	
\subfile{e3_tp2_ej2.tex}
\let\input=\oldinput

\let\oldinput=\input 	
\def\input#1{\oldinput{Ej3/Informe/#1}}	
\subfile{e3_tp2_ej3.tex}
\let\input=\oldinput

\let\oldinput=\input 	
\def\input#1{\oldinput{Ej4/Informe/#1}}	
\subfile{e3_tp2_ej4.tex}
\let\input=\oldinput

\let\oldinput=\input 	
\def\input#1{\oldinput{Ej5/Informe/#1}}	
\subfile{e3_tp2_ej5.tex}
\let\input=\oldinput


\let\oldinput=\input 	
\def\input#1{\oldinput{Ej6/Informe/#1}}	
\subfile{e3_tp2_ej6.tex}
\let\input=\oldinput


\let\oldinput=\input 	
\def\input#1{\oldinput{Ej7/Informe/#1}}	
\subfile{e3_tp2_ej7.tex}
\let\input=\oldinput



\let\oldinput=\input 	
\def\input#1{\oldinput{Ej8/Informe/#1}}	
\subfile{e3_tp2_ej8.tex}
\let\input=\oldinput

\end{multicols}

%\subfile{e3_tp2_anexo.tex}
\end{document}
